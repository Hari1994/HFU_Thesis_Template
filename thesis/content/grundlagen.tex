\chapter{Grundlagen}

\section{Kapitel1}

Dieser Text ist von Nobody ~\cite{article}
Dieser Text ist auch von Nobody \cite{book}
\subsection{Unterkapitel}


\begin{figure}[htb!]
	\caption{Beispiel Bild}
	\includegraphics[width=0.9\textwidth]{content/pictures/hfu.jpg}
	\label{pic:label}
\end{figure}


% longtable ist eine schöne Form der Tabelle
% Über die Kopfzeile "\begin{longtable}[l]{|p{3,0cm}|p{10,5cm}|}" kann man die Anzahl der
% Spalten bestimmten. Man fügt einfach "|" ein und schreibt dazwischen, wie breit die Spalte sein % soll. Die Breite gibt an typischerweise über "p{X.Ycm}" an, wobei X und Y ganze Zahlen sind.
% Es ist auch möglich zwei senkrechte Striche innerhalb der Tabelle zu machen, indem mann "||" setzt. Oder zwei vertikale Striche, indem man zwei Mal "\hline" setzt.
% Innerhalb der Tabelle macht ein "\hline", um eine neue Zeile einzuleiten - Wichtig ist, dass
% vor dem \hline ein Zeilenumbruch in Form von "\\" kommt, da ansonsten ein Fehler ausgegeben wird.
\begin{footnotesize}
	\begin{longtable}[l]{|p{3,0cm}|p{10,5cm}|}
		\caption{Beispiel schöne Tabelle}
		\label{tab:label}
		\\
		\hline
		Spalte1/Zeile1	& 
		\begin{itemize}
			\item Itemize in Tabelle
			\item Spalte2
		\end{itemize} 
		\\ 
		\hline
		Spalte1/Zeile2	&   
		Spalte2/Zeile2
		\\
		\hline
		Spalte1/Zeile3	&    
		Spalte2/Zeile3
		\\
		\hline
	\end{longtable}
\end{footnotesize}

